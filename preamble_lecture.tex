%!TEX root = lectures.tex

% Start indexing
\makeindex

% Make headings
\pagestyle{headings}

% Remove the section number from the section name in headers.
\renewcommand{\sectionmark}[1]{\uppercase{\markboth{#1}{#1}}}


% Label sections as 'lecture 1', etc.
\titleformat{\section}{\normalfont\Large\bfseries}{Lecture \thesection}{1em}{}

% Same in table of contents.
\renewcommand{\cftsecpresnum}{Lecture\space}
\newlength\lecturelength
\settowidth\lecturelength{\cftsecpresnum\space\space}
\addtolength\cftsecnumwidth{\lecturelength}

\setlength\intextsep{0pt}

% Change name of table of contents
\renewcommand{\contentsname}{Table of Contents}

% Remove the Abstract title from the abstract.
\renewcommand{\abstractname}{}    % Clear the title
\renewcommand{\absnamepos}{empty} % Originally centre

% Number all equations and figures within subsections.
\numberwithin{equation}{subsection}
\numberwithin{figure}{subsection}

% Define some fake (i) label macros.
\newcommand{\fakeitemref}[1]{% \fakeitem{<n>}
	{\upshape(\emph{\romannumeral 0#1})}}
\newcommand{\fakeitem}[1]{% \fakeitem{<n>}
	\fakeitemref{#1}~}

% Make text both bold and emphasised, for keywords in definitions and similar.
\newcommand{\keyword}[1]{% \keyword{text}
	{\textbf{\emph{#1}}}}

% Roman numerals
\makeatletter
\newcommand*{\rom}[1]{\expandafter\@slowromancap\romannumeral #1@}
\makeatother


% Alias for \section with optional footnote for date.
\newcommand{\lecture}[2][]{% \lecture[Date]{Title}
	%\section[#2]{#2\ifblank{#1}{}{\blfootnote{Date: #1.}}}%
	\section{#2}%
}

% Alias for \subsection that is more semantic.
\newcommand{\topic}{\subsection}

% Aliases for enumerate with certain label settings preapplied.
\newenvironment{romanlist}{\begin{enumerate}[label = \textup{(\emph{\roman*})}]}{\end{enumerate}}
\newenvironment{alphalist}{\begin{enumerate}[label = \textup{(\alph*)}]}{\end{enumerate}}

% Define a footnote macro that doesn't leave a mark.
\makeatletter
\def\blfootnote{\gdef\@thefnmark{}\@footnotetext}
\makeatother
