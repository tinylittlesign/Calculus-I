%!TEX root = ../lectures.tex

An important type of functions (their importance will be justified throughout the course) are ones that, in some sense, don't jump.
That is to say, when we draw their graph on a piece of paper, we (sort of) never need to lift the pen.

However it isn't entirely straight forward how to define this.
In order to move toward a good definition, we'll discuss two important special cases that cover most, but not all, of our needs.

\topic{Special Cases}

\subsubsection*{The domain is an open interval}

If a function $f$ has the domain
\[
	X = \Set{x \in \R \given b < x < c},
\]
for some fixed $b$ and $c$, then $f$ is said to be \keyword{continuous}\index{continuity} if and only if for all $a \in X$ we have
\[
	\lim_{x \to a} f(x) = f(a).
\]

\noindent
In other words, the function agrees with its limit at every point.

\subsubsection*{The domain is a closed interval}

If a function $f$ has the domain
\[
	X = \Set{x \in \R \given b \leq x \leq c},
\]
where $b$ and $c$ are some constants, then $f$ is said to be \keyword{continuous} if and only if the following are satisfied:
\begin{romanlist}
	\item for all interior points, i.e. $b < a < c$, we have $\lim\limits_{x \to a} f(x) = f(a)$,
	\item for the left endpoint $b$, we have $\lim\limits_{x \to b^+} f(x) = f(b)$, and
	\item for the right endpoint $c$, we have $\lim\limits_{x \to c^-} f(x) = f(c)$.
\end{romanlist}

\noindent
The reason for this distinction is hopefully intuitively clear: if we are at an endpoint of the closed interval, we can't hope to find a deleted two-sided neighbourhood around this point for which the function is defined, and therefore by definition it cannot have a two-sided limit at this point.
Even so we still want to have a notion of continuity at such a point, so effectively what we do is the following:

We investigate the continuity of a function point by point, in the way that is reasonable depending on the point and its surrounding:
\begin{itemize}
	\item If the function is defined in a two-sided neighbourhood around a point, we require that the two-sided limit exists and agrees with the function value.
	\item If the function is defined only in a right neighbourhood of the point, we (obviously?) require only that the right limit exists and equals the function value.
	\item Likewise is the function is defined only in a left neighbourhood, we require a left limit equal to the value of the function.
\end{itemize}

\noindent
Of course mathematicians don't like special cases if they can avoid them, so we attempt to construct a general definition motivated by this discussion.

\topic{Definition of Continuity}

\begin{definition}[Continuity in a point]
	We say that the function $f \colon X \to Y$ is \keyword{continuous in the point}\index{continuity}\index{continuity!point} $a \in \R$ if and only if the following two conditions are satisfied:
	\begin{romanlist}
		\item $a \in X$, and
		\item for each $\varepsilon > 0$ there exists a $\delta > 0$ such that, for all $x \in X$, we have that if $\abs{x - a} < \delta$, then $\abs{f(x) - f(a)} < \varepsilon$.
	\end{romanlist}
\end{definition}

\begin{remark}
	There is a crucial difference in this epsilon-delta construction compared to that of our limit definition, namely that we don't actually demand that the function is defined anywhere except for the point $a$ itself. However if the function happens to be defined outside of this point $a$, then we impose some restrictions on these exterior points!
	It is this distinction that captures the subtleties of the second special case above.

	One consequence of this is that a function defined in an isolated point, say $f \colon \Set{1} \to \Set{2}$, defined by $f(1) = 2$, is continuous at that point $x = 1$ (try to prove it using the above defintion!). This might seem strange, but it turns out to be a good property to have, that moreover agrees with much more abstract definitions of continuity found in the field of topology.
\end{remark}

\noindent
Another interesting takeaway from this definition is that in principle a function is \keyword{discontinuous}\index{discontinuity} (not continuous) in any point not in its domain. However, when we study a function as a whole, we will consider only points in the domain:

\begin{definition}[Continuity of a function]
	We say that the function $f \colon X \to Y$ is a \keyword{continuous function}\index{continuity!function} if and only if the function is continuous for all $x \in X$.
\end{definition}

\noindent
This means that there are functions which we call continuous, despite actually being discontinuous in one or more point; it's just that these points are outside of the domain!

\begin{figure}
	\centering
	\begin{tikzpicture}
		\begin{axis}[
			scale = 1,
			axis x line = middle,
			axis y line = middle,
			xmin = -2,
			xmax = 12,
			ymin = -10,
			ymax = 10,
			xlabel = $x$,
			ylabel = $y$,
			every axis x label/.style = {
				at = {(ticklabel* cs:1.01)},
				anchor = west,
			},
			every axis y label/.style = {
				at = {(ticklabel* cs:1.01)},
				anchor = south,
			},
			]
			\addplot[
			black,
			samples = 100,
			domain = -2:6.9,
			]{1/(x-7)};
			\addplot[
			black,
			samples = 100,
			domain = 7.1:12,
			]{1/(x-7)};
		\end{axis}
	\end{tikzpicture}
	\caption{Plot of $\displaystyle f(x) = \frac{1}{x - 7}$.}
\end{figure}

\begin{example}
	The function $f \colon \R \setminus \Set{7} \to \R$ defined by
	\[
		f(x) = \frac{1}{x - 7}
	\]
	is discontinuous at $x = 7$, but continuous for all other $x \in \R$.
	But because $x = 7$ doesn't belong to the domain, $f$ is a continuous function anyway!
\end{example}

\noindent
We started off talking about the graph of a continuous function sort of not having a jump.
Clearly, this isn't entirely true, as evidenced by this example.
So with what we now know in hand, a better (and also correct) characterisation is that a continuous function has a graph that has no jumps in any interval in its domain.

\topic{Continuous Functions}

Many functions we are familiar with are continuous.

\begin{examples}
	All polynomials are continuous, as are all rational functions, and all rational powers $x^{m/n}$, $m, n \in \Z$, $n \neq 0$.

	The trigonometric functions (sine, cosine, tangent, etc.) are all continuous.

	The absolute value $\abs{x}$ is continuous as well.
\end{examples}

\noindent
We can also combine continuous functions to get new continuous functions:

\begin{theorem}[Combining continuous functions]
	If both $f$ and $g$ are functions defined on some neighbourhood containing $c$ and both are continuous at $c$, then the following functions are continuous at $c$:
	\begin{romanlist}
		\item $f + g$ and $f - g$;
		\item $fg$;
		\item $k f$, where $k \in \R$ is a constant;
		\item $\displaystyle \frac{f}{g}$, provided $g(c) \neq 0$;
		\item $(f)^{1 / n}$, $n \in \Z$, provided $f(c) > 0$ if $n$ is even.
	\end{romanlist}
\end{theorem}

\begin{proof}
	We simply use the rules for calculating limits from the last lecture.
	For instance:
	\begin{align*}
		\lim_{x \to c} (f \pm g)(x) & = \lim_{x \to c} \big ( f(x) \pm g(x) \big ) = \lim_{x \to c} f(x) \pm \lim_{x \to c} g(x) \\
		                            & = f(c) \pm g(c) = (f \pm g)(c)
	\end{align*}
	and
	\[
		\lim_{x \to c} (fg)(x) = \lim_{x \to c} \big ( f(x)g(x) \big ) = \big ( \lim_{x \to c} f(x) \big ) \big ( \lim_{x \to c} g(x) \big ) = f(c) g(c) = (fg)(c).
	\]

	\noindent
	Note that for endpoints of closed intervals in the domains of the functions, we would of course have to replace the above by appropriate one-sided limits.
\end{proof}

\noindent
A trickier result to prove, but one that is extremely useful is the following:

\begin{theorem}
	If $f(g(x))$ is defined on some neighbourhood of $c$, and if $f$ is continuous at $L$, with
	\[
		\lim_{x \to c} g(x) = L,
	\]
	then
	\[
		\lim_{x \to c} f(g(x)) = f(L) = f \big ( \lim_{x \to c} g(x) \big ).
	\]

	\noindent
	Furthermore, if $g$ is continuous at $c$ (i.e. $L = g(c)$), then $f \circ g$ is continuous at $c$ as well:
	\[
		\lim_{x \to c} f(g(x)) = f(g(c)).
	\]
\end{theorem}

\begin{proof}
	Let $\varepsilon > 0$.
	Because we will need to keep track of whether we're in the domain of $f$ or $g$, we will denote them by $X_f$ and $X_g$, respectively.
	Further we use $X_{f \circ g}$ to refer to the domain of the composition.
	We now establish what we know from the assumptions in the theorem.

	Since $f$ is continuous at $L = \lim\limits_{x \to c} g(x)$, there exists some $\delta_1 > 0$ such that for all $y \in X_f$ with
	\[
		\abs[\big]{y - \underbrace{\lim_{x \to c} g(x)}_{= L}} < \delta_1,
	\]
	we have that
	\begin{equation}\label{lec3:fcontinuous}
		\abs[\Big]{f(y) - f \big ( \lim_{x \to c} g(x) \big ) } < \varepsilon.
	\end{equation}

	\noindent
	Moreover since $\lim\limits_{x \to c} g(x) = L$, there exists some $\delta_2 > 0$ such that for all $x \in X_g$ with $\abs{x - c} < \delta_2$ we have $\abs{g(x) - L} < \delta_1$.

	We take a moment to point out what happened in the very last step there: in the definition of the limit of $g$, we are able to take \emph{any} $\varepsilon$, so in particular we can fix it to a specific value, namely $\delta_1$, and find a corresponding $\delta_2$ which makes the definition work.

	If we now take $x \in X_{f \circ g}$ such that $\abs{x - c} < \delta_2$, then $\abs{g(x) - L} < \delta_1$, so by the construction of $\delta_1$, Equation \eqref{lec3:fcontinuous} with $y = g(x)$ becomes
	\[
		\abs{f(g(x)) - f(L)} = \abs[\Big]{f(g(x)) - f \big ( \lim_{x \to c} g(x) \big )} < \varepsilon,
	\]
	so $f \circ g$ is continuous at $x = c$.
\end{proof}

\noindent
This means that not only are many of the elementary functions we are used to continuous, but almost any conceivable combination of them are as well!

\topic{Removable Discontinuities}

Recalling how functions are by definition discontinuous at any point where they are undefined, it might sometimes be possible to therefore define a function at some point where it wasn't previously defined, in such a way that the new function is continuous.
We demonstrate this by example:

\begin{example}
	Recall the function
	\[
		f(x) = \frac{x^2 - 1}{x - 1}
	\]
	from Example \ref{lec2:linewithhole} in the previous lecture.
	It is not continuous over all of $\R$, since it isn't even defined in $x = 1$, but it can be redefined to become continuous everywhere.

	Since $\lim\limits_{x \to 1} f(x) = 2$, we \emph{define} a new function $F$ which is $F(x) = f(x)$ for all $x \neq 1$, and $F(1) = 2$, so
	\[
		F(x) = \begin{cases}
		\frac{x^2 - 1}{x - 1}, & \text{if}~ x \neq 1 \\
		2, & \text{if}~ x = 1
		\end{cases}
		= x + 1,
	\]
	which is continuous everywhere.
	In this case, $x = 1$ is said to be a \keyword{removable discontinuity}\index{discontinuity!removable}, and $F(x)$ is said to be a \keyword{continuous extension}\index{continuity!extension} of $f(x)$.
\end{example}

\topic{Continuous Functions on Closed, Finite Intervals}

Continuous functions defined on closed and finite intervals have a few special properties.

\begin{theorem}[Max-min theorem]\index{max-min theorem}
	If $f$ is continuous on the closed finite interval $[a, b]$, then there exists $p, q \in [a, b]$ such that for all $x \in [a, b]$ we have
	\[
		f(p) \leq f(x) \leq f(q).
	\]

	\noindent
	In other words, $f(p)$ is an global minimum of the function, attained on the point $x = p$, and $f(q)$ is an global maximum, at the point $x = q$.
\end{theorem}

\begin{figure}
	\centering
	\begin{tikzpicture}
		\begin{axis}[
			scale = 1,
			axis x line = middle,
			axis y line = middle,
			xmin = -1,
			xmax = 4,
			ymin = -2,
			ymax = 5,
			xlabel = $x$,
			ylabel = $y$,
			every axis x label/.style = {
				at = {(ticklabel* cs:1.01)},
				anchor = west,
			},
			every axis y label/.style = {
				at = {(ticklabel* cs:1.01)},
				anchor = south,
			},
			]
			\addplot[
			black,
			samples = 100,
			domain = 0.2:3,
			]{3+(x-3)^2*sin((x-3)r)};
			\addplot[mark=*] coordinates {(3, 3)} node[pin=150:{$f(q)$}]{} ;
			\addplot[mark=*] coordinates {(0.711, -0.945)} node[pin=-30:{$f(p)$}]{} ;
		\end{axis}
	\end{tikzpicture}
	\caption{Example of the Max-min theorem.}
\end{figure}

\noindent
A related result (it turns out) is the following:

\begin{theorem}[Intermediate value theorem]\index{intermediate value theorem}
	If $f$ is continuous on the interval $[a, b]$, with $a < b$ being some constants, and if $s$ is a number between $f(a)$ and $f(b)$, then there exists a number $c \in [a, b]$ such that $f(c) = s$.
\end{theorem}

\noindent
We demonstrate their usefulness by a brief example.

\begin{example}
	Show that $f(x) = x^3 - 4x + 1$ has a root in the interval $[1, 2]$.

	Note first that $f(1) = 1^3 - 4 (1) + 1 = 1 - 4 + 1 = -2$, and $f(2) = 2^3 - 4(2) + 1 = 8 - 8 + 1 = 1$.
	Moreover, since $-2 \leq 0 \leq 1$, and since $f$ is continuous (since all polynomials are continuous everywhere), by the intermediate value theorem there must exist some $c \in [1, 2]$ such that $f(c) = 0$.

	Note that this tells us that a root of the polynomial exists in this interval, but it doesn't tell us what exactly it is!
\end{example}

\subsubsection*{Some Notes on Proofs}

Both the Max-min theorem and the Intermediate value theorem seem quite obvious, but actually proving them require some deep insights into upper bounds of sets of real numbers and axiomatic details to do with real numbers. It is beyond the scope of this course, but if we have some spare time and if there is interest we might prove them anyway.
The curious student may find details in \cite[Appendix~\rom{3}]{Adams2013}.
