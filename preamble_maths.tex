%!TEX root = lectures.tex

% Define some common variously styled letters we occasionally need.
\newcommand{\C}{\ensuremath{\mathbb{C}}}
\newcommand{\Q}{\ensuremath{\mathbb{Q}}}
\newcommand{\R}{\ensuremath{\mathbb{R}}}
\newcommand{\Z}{\ensuremath{\mathbb{Z}}}
\newcommand{\N}{\ensuremath{\mathbb{N}}}

% Allow overwriting of \Re and \Im.
\let\Re\relax
\let\Im\relax
% Redefine \Re and \Im as Re and Im instead of mathfrak letters.
\DeclareMathOperator{\Re}{Re}
\DeclareMathOperator{\Im}{Im}

% Declare an image symbol
\DeclareMathOperator{\im}{im}

% Define a placeholder for functions without arguments.
\newcommand*{\placeholder}{\makebox[1ex]{\textbf{$\cdot$}}}


% Define a plethora of theorem/definition/lemma/etc. environments.
% Number them all using the same counter.
\newtheorem{theorem}{Theorem}[subsection]
\newtheorem*{theorem*}{Theorem}
\newtheorem{corollary}[theorem]{Corollary}
\newtheorem{lemma}[theorem]{Lemma}
\newtheorem{proposition}[theorem]{Proposition}
\newtheorem{conjecture}[theorem]{Conjecture}
\theoremstyle{definition}
\newtheorem{definition}[theorem]{Definition}
\newtheorem{definitions}[theorem]{Definitions}
\newtheorem{notation}[theorem]{Notation}
\newtheorem{axiom}[theorem]{Axiom}
\newtheorem{exercise}[theorem]{Exercise}
\newtheorem{exercises}[theorem]{Exercises}
\newtheorem{problem}[theorem]{Problem}
\theoremstyle{remark}
\newtheorem{remark}[theorem]{Remark}


% Define special example(s) and solution(s) environments in order to have end-of-environment marks.
\declaretheorem[
	style = definition,
	qed = $\blacktriangle$,
	sibling = theorem
]{example}
\declaretheorem[
	style = definition,
	qed = $\blacktriangle$,
	sibling = theorem
]{examples}

\declaretheorem[
	style = definition,
	qed = $\blacktriangle$,
	sibling = theorem
]{counterexample}

\declaretheorem[
	style = remark,
	qed = $\blacklozenge$,
	numbered = no
]{solution}
\declaretheorem[
	style = remark,
	qed = $\blacklozenge$,
	numbered = no
]{solutions}




% Redefine \overline to the nicely semantic \conjugate.
\newcommand*{\conjugate}[1]{\overline{#1}}



% For all of the below, use \command* to make the delimiters adjust size automatically.

% Defines an absolute value notation. Give it no argument and it'll default to \abs{\placeholder}.
\DeclarePairedDelimiterX{\abs}[1]{\lvert}{\rvert}{% \abs{a}
	\ifblank{#1}{\placeholder}{#1}%
}

% Provide semantic notation for describing sets with conditions.
\providecommand\given{} % Just make sure this exists, so that things don't explode.
\DeclarePairedDelimiterX{\Set}[1]{\{}{\}}{% \Set{ x \given x > 0}
	\,\renewcommand\given{\nonscript\:\delimsize\vert\nonscript\:\mathopen{}}#1\,
}

% Set the interval package to use parens for open intervals.
\intervalconfig{
	soft open fences
}
